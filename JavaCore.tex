% XeLaTex源码模板
\documentclass[12pt, a4paper]{article}
\usepackage{xeCJK}
\setCJKmainfont{WenQuanYi Micro Hei}
\usepackage{graphicx}
\parindent 2em   %段首缩进


\begin{document}

\title{\bf JavaCore~\LaTeX~}
\author{Tqcenglish \\ \it tqcenglish@gmail.com}
\date{\today}
\maketitle
%设置目录
\tableofcontents
\setcounter{tocdepth}{2}
\newpage
%设置摘要
\begin{abstract}
    关于Java核心技术的学习
\end{abstract}
%换页
\newpage
\section{泛型的意义}
泛型比使用杂乱的Object变量进行强制转换更具有安全性和可读性。
\subsection{为什么使用泛型}
泛型程序设计(Generic programming)意味着编写的代码可以被很多不同类型的对象所重用。
类型参数(type parameters)可以避免强制转换异常。泛型程序员需要预测出类未来可能有的所有用途。
%设置原文照排
\begin{verbatim}
\end{verbatim}

\end{document}
